\documentclass[11pt,letterpaper,twocolumn]{article}
\usepackage[english,brazilian]{babel}
\usepackage[utf8]{inputenc}
\usepackage[T1]{fontenc}
\usepackage{float}
\usepackage{xcolor}
\usepackage{verbatim}
\usepackage{charter}
\usepackage{amsmath}
\usepackage{appendix}
\usepackage{ragged2e}
\usepackage{array}
\usepackage{etoolbox}
\usepackage{fancyhdr}
\usepackage{booktabs}
\usepackage{arydshln}
\usepackage{caption}
\usepackage{subcaption}
\usepackage{enumitem}
\usepackage{geometry}
\geometry{
  top=0.8in,            
  inner=0.5in,
  outer=0.5in,
  bottom=0.9in,
  headheight=4ex,       
  headsep=6.5ex,         
}
\usepackage{graphicx}
\usepackage{mathtools}
\usepackage{multirow}
\usepackage{pdfpages}
\usepackage{subfiles}
\usepackage[compact]{titlesec}
\usepackage{stfloats}
\usepackage{natbib,stfloats}
%\usepackage[style=abnt]{biblatex}


\setlength{\columnsep}{30pt}

\pagestyle{fancy}
\fancyhf{}
      
\fancyfoot{}
\fancyfoot[C]{\thepage} % page
\renewcommand{\headrulewidth}{0mm} % headrule width
\renewcommand{\footrulewidth}{0mm} % footrule width

\makeatletter
\patchcmd{\headrule}{\hrule}{\color{black}\hrule}{}{} % headrule
\patchcmd{\footrule}{\hrule}{\color{black}\hrule}{}{} % footrule
\makeatother

\definecolor{blueM}{cmyk}{1.0,0.49,0.0,0.47}

\chead[C]{
      \begin{tabular}{m{1.6cm}m{11.4cm}m{2.5cm}}
      \includegraphics[height=1.5cm]{images/Einstein.png} 
      &
      \centering
     \fcolorbox{white}{blueM}{\fbox{\begin{minipage}{11.5cm}
     \centering
     \textcolor{white}{Mad Scientists Diary}
     \end{minipage}}}
         &
        \centering
         \tiny{ \vspace{3.5mm} GAMOS\\%Para usar en otras carreras consulte a su coordinador
%%%%%%%%%%%%%%%%%%%%%%%%%%%%%%%%%%%%%%%%%%%%%%%%%%%%%%%%%%%%%%%%%%%%%%%%%%%%%%%%%%%%%%%%%%%%%%%%%
          publication of works in progress\\ %Elija el ciclo que corresponda
%%%%%%%%%%%%%%%%%%%%%%%%%%%%%%%%%%%%%%%%%%%%%%%%%%%%%%%%%%%%%%%%%%%%%%%%%%%%%%%
          }\tabularnewline
%          \hline
          \end{tabular}%
    }
    
\begin{document}
\twocolumn[\begin{@twocolumnfalse}


%\begin{minipage}{0.15\textwidth}{
%    \includegraphics[width=4cm]{images/Einstein.png}}
%\end{minipage}
\hspace{25pt}
\begin{minipage}{0.90\textwidth}
\vspace{5mm}
    \center\Large{\textbf{Identificão de padrões de demanda para micro e pequenas empresas do Agreste Pernambucano}} 
    \vspace{3mm}
    
    \large{\textbf{Tatiana Balbi Fraga}} 
    \vspace{2mm}
    
    %\large{\textbf{Asesor 1: Nombre del Asesor$^1$} ; Asesor 2: Nombre del Asesor$^2$} \newline
    %Si solo hay un asesor borrar el ``1''
    \fontsize{0.35cm}{0.5cm}\selectfont \textit{Núcleo de Tecnologia, Centro Acadêmico do Agreste, Universidade Federal de Pernambuco}
    \vspace{1mm} 
    
    \today % FECHA

\end{minipage}

\small

\vspace{11pt}

\centerline{\rule{0.95\textwidth}{0.4pt}}

\begin{center}
    
    \begin{minipage}{0.9\textwidth}
        % RESUMEN
        \noindent \textbf{Abstract:} Este artigo trata do texto elaborado para projetos de pesquisa a serem desenvolvidos por alunos e professores da equipe GAMOS. Tais projetos serão desenvolvidos em diferentes empresas do Agreste Pernambucano, como parte do Projeto de Pesquisa 'Estudo de Métodos de Previsão de Demanda e Proposição de Metodologia Combinada no Contexto das Micro e Pequenas Empresas do Agreste Pernambucano', especialmente na forma de projetos de iniciação científica ou projetos de conclusão de curso. 
    
        \vspace{4mm}
        % PALABRAS CLAVE
        \noindent \textbf{Key words:} Pradrões de Demanda, Micro e Pequenas Empresas, Projetos de Pesquisa.
    
    \end{minipage}
    
\end{center}

\centerline{\rule{0.95\textwidth}{0.4pt}}

\vspace{15pt}
\end{@twocolumnfalse}]
%%%%%%%%%%%%%%%%%%%%%%%%%%%%%%%%%%%%%%%%%%%%%%%%%%%%%%%%%%%%

\section{Introdução}

A previsão de demanda é essencial para o bom planejamento em qualquer empresa. Através de uma boa previsão é possível, entre outras coisas, controlar melhor os níveis de estoque, reduzindo custos e oferecendo um melhor nível de serviço aos clientes. 

Conforme mostra \cite{MakridakisHibon2000}, a literatura apresenta uma grande variedade de metodologias que podem ser utilizadas para previões de demanda, sendo que a performance dos distintos modelos de previsão varia de acordo a natureza dos dados e um modelo que gera bons resultados para determinada classe de itens de uma empresa pode gerar previsões ruins para outros itens dessa mesma empresa.

Uma estratégia natural utilizada para identificar o modelo de previsão adequado pra cada item consiste em comparar a performance dos distintos modelos candidatos utilizando dados históricos de vendas do item \citep{UlrichEtAl2022}. Contudo, como geralmente as empresas produzem e/ou comercializam uma grande variedade de itens, essa estratégia acaba se tornando um esforço considerável. 

De acordo com \cite{UlrichEtAl2022}, uma opção viável consiste em agrupar os itens de acordo com seus padrões de demanda, para posteriormente identificar o modelo de previsão adequado pra cada grupo e não mais para cada item individual. O projeto elaborado por \cite{Fraga2019} propõem uma abordagem similar, contudo busca a identificação de padrões de demanda para os principais itens produzidos e/ou comercializados por um grupo de micro e pequenas empresas do Agreste Pernambucano, visando o desenvolvimento de uma metodologia combinada que seja adequada a um cojunto de padrões de demanda distintos e recorrentes nestas empresas. Como parte do projeto proposto por \cite{Fraga2019}, o GAMOS estará desenvolvendo subprojetos de pesquisa, buscando identitificar os padrões de demanda dos principais produtos de empresas de diferentes setores da região. Cada subprojeto buscará o atingimento dos seguintes objetivos:

\begin{itemize}
  \item identificação de metodologia atualmente aplicada para previsão de demanda na empresa;
  \item identificação dos principais produtos; 
  \item levantamento de dados (históricos de vendas e outros dados relevantes);
  \item compreensão de metodologias aplicadas para identificação de padrão de demanda; e
  \item identificação dos padrões de demanda.
\end{itemize}
  
Os dados coletados nestes projetos, assim como os estudos realizados serão de grande importância científica, tendo em vista que serão utilizados para o desenvolvimento de uma nova metodologia de previsão de demanda combinada, e também poderão ser utilizados para outros trabalhos futuros relacionados aos setores estudados, tornando se referência para diversos estudos que venham a ser desenvolvidos.

\section{Fundamentação Teórica}

De acordo com \cite{BoylanEtAl2008}, os padrões de demanda não normais podem ser classificados da seguinte forma:

\begin{itemize}
  \item um item de \emph{demanda intermitente} é um item com ocorrências de demanda pouco frequêntes;
  \item um item de \emph{demanda de movimento lento} é um item cuja demanda média por período é baixa. Isso pode ser devido a ocorrências de demanda pouco frequentes, tamanhos médios de demanda baixos ou ambos;
  \item um item de \emph{demanda errática} é um item cujo tamanho de demanda é altamente variável;
  \item um item de \emph{demanda grumosa} é um item intermitente para o qual a demanda, quando ocorre, é altamente variável.; e
  \item um item de \emph{demanda agregada} é um item intermitente para o qual a demanda, quando ocorre, é constante (ou quase constante).
\end{itemize}

\begin{thebibliography}{9}

\bibitem[\protect\citeauthoryear{Ballou}{2001}]{Ballou2001}
Ballou, R.H. (2001).{\it Gerenciamento da Cadeia de Suprimentos: Planejamento, Organização e Logística Empresarial}, 4. ed., Porto Alegre: Bookman.

\bibitem[\protect\citeauthoryear{Boylan et al}{2008}]{BoylanEtAl2008}
Boylan, J.E., Syntetos, A.A., e Karakostas, G.C. (2008). 'Classification for forecasting and stock control:a case study'. {\it Journal of the Operational Research Society}, Vol. 59, pp. 473--481.

\bibitem[\protect\citeauthoryear{Fraga}{2019}]{Fraga2019}
Fraga, T.B. (2019). 'Estudo de Métodos de Previsão de Demanda e Proposição de Metodologia Combinada no Contexto das Micro e Pequenas
Empresas do Agreste Pernambucano'. Projeto de Pesquisa registrado em 09/11/2019, e aprovado pela Pró-reitoria de Pesquisa da UFPE em 11/02/2021 (Processo SIPAC 23076.057489/2019-21).

\bibitem[\protect\citeauthoryear{Makridakis e Hibon}{2000}]{MakridakisHibon2000}
Makridakis, S. e Hibon, M. (2000) 'The M3-Competition: results, conclusions and implications'. {\it International Journal of Forecasting}, Vol. 16, pp. 451--476.

\bibitem[\protect\citeauthoryear{Ulrich et al.}{2022}]{UlrichEtAl2022}
Ulrich, M., Jahnke, H., Langrock, R., Pesch, R., e Senge, R. (2022) 'Classification-based model selection in retail demand forecasting'. {\it International Journal of Forecasting}, Vol. 38 (1), pp. 209--223.

\bibitem[\protect\citeauthoryear{Werner et al.}{2006}]{WernerEtAl2006}
Werner, L, Lemos, F.O., Daudt, T. (2006) 'Previsão de demanda e níveis de estoque uma abordagem conjunta aplicada no setor siderúrgico'. {\it XIII SIMPEP}, Bauru, SP, Brasil.


\end{thebibliography} 

\end{document}

