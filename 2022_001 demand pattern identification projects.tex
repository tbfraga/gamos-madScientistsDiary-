\documentclass[11pt,letterpaper,twocolumn]{article}
\usepackage[english,brazilian]{babel}
\usepackage[utf8]{inputenc}
\usepackage[T1]{fontenc}
\usepackage{float}
\usepackage{xcolor}
\usepackage{verbatim}
\usepackage{charter}
\usepackage{amsmath}
\usepackage{appendix}
\usepackage{ragged2e}
\usepackage{array}
\usepackage{etoolbox}
\usepackage{fancyhdr}
\usepackage{booktabs}
\usepackage{arydshln}
\usepackage{caption}
\usepackage{subcaption}
\usepackage{enumitem}
\usepackage{geometry}
\geometry{
  top=0.8in,            
  inner=0.5in,
  outer=0.5in,
  bottom=0.9in,
  headheight=4ex,       
  headsep=6.5ex,         
}
\usepackage{graphicx}
\usepackage{mathtools}
\usepackage{multirow}
\usepackage{pdfpages}
\usepackage{subfiles}
\usepackage[compact]{titlesec}
\usepackage{stfloats}
\usepackage{natbib,stfloats}
%\usepackage[style=abnt]{biblatex}

\setlength{\columnsep}{30pt}

\pagestyle{fancy}
\fancyhf{}
      
\fancyfoot{}
\fancyfoot[C]{\thepage} % page
\renewcommand{\headrulewidth}{0mm} % headrule width
\renewcommand{\footrulewidth}{0mm} % footrule width

\makeatletter
\patchcmd{\headrule}{\hrule}{\color{black}\hrule}{}{} % headrule
\patchcmd{\footrule}{\hrule}{\color{black}\hrule}{}{} % footrule
\makeatother

\definecolor{blueM}{cmyk}{1.0,0.49,0.0,0.47}

\chead[C]{
      \begin{tabular}{m{1.6cm}m{11.4cm}m{2.5cm}}
      \includegraphics[height=1.5cm]{images/Einstein.png} 
      &
      \centering
     \fcolorbox{white}{blueM}{\fbox{\begin{minipage}{11.5cm}
     \centering
     \textcolor{white}{Mad Scientists Diary}
     \end{minipage}}}
         &
        \centering
         \tiny{ \vspace{3.5mm} GAMOS\\%Para usar en otras carreras consulte a su coordinador
%%%%%%%%%%%%%%%%%%%%%%%%%%%%%%%%%%%%%%%%%%%%%%%%%%%%%%%%%%%%%%%%%%%%%%%%%%%%%%%%%%%%%%%%%%%%%%%%%
          publication of works in progress\\ %Elija el ciclo que corresponda
%%%%%%%%%%%%%%%%%%%%%%%%%%%%%%%%%%%%%%%%%%%%%%%%%%%%%%%%%%%%%%%%%%%%%%%%%%%%%%%
          }\tabularnewline
%          \hline
          \end{tabular}%
    }
    
\begin{document}
\twocolumn[\begin{@twocolumnfalse}


%\begin{minipage}{0.15\textwidth}{
%    \includegraphics[width=4cm]{images/Einstein.png}}
%\end{minipage}
\hspace{25pt}
\begin{minipage}{0.90\textwidth}
\vspace{5mm}
    \center\Large{\textbf{Identificão de padrões de demanda para micro e pequenas empresas do Agreste Pernambucano}} 
    \vspace{3mm}
    
    \large{\textbf{Tatiana Balbi Fraga}} 
    \vspace{2mm}
    
    %\large{\textbf{Asesor 1: Nombre del Asesor$^1$} ; Asesor 2: Nombre del Asesor$^2$} \newline
    %Si solo hay un asesor borrar el ``1''
    \fontsize{0.35cm}{0.5cm}\selectfont \textit{Núcleo de Tecnologia, Centro Acadêmico do Agreste, Universidade Federal de Pernambuco}
    \vspace{1mm} 
    
    \today % FECHA

\end{minipage}

\small

\vspace{11pt}

\centerline{\rule{0.95\textwidth}{0.4pt}}

\begin{center}
    
    \begin{minipage}{0.9\textwidth}
        % RESUMEN
        \noindent \textbf{Abstract:} Este artigo trata do texto elaborado para projetos de pesquisa a serem desenvolvidos por alunos e professores da equipe GAMOS. Tais projetos serão desenvolvidos em diferentes empresas do Agreste Pernambucano, como parte do Projeto de Pesquisa 'Estudo de Métodos de Previsão de Demanda e Proposição de Metodologia Combinada no Contexto das Micro e Pequenas Empresas do Agreste Pernambucano', especialmente na forma de projetos de iniciação científica ou projetos de conclusão de curso. 
    
        \vspace{4mm}
        % PALABRAS CLAVE
        \noindent \textbf{Key words:} Pradrões de Demanda, Micro e Pequenas Empresas, Projetos de Pesquisa.
    
    \end{minipage}
    
\end{center}

\centerline{\rule{0.95\textwidth}{0.4pt}}

\vspace{15pt}
\end{@twocolumnfalse}]
%%%%%%%%%%%%%%%%%%%%%%%%%%%%%%%%%%%%%%%%%%%%%%%%%%%%%%%%%%%%

\section{Introdução}

A previsão de demanda é essencial para o bom planejamento em qualquer empresa. Através de uma boa previsão é possível, entre outras coisas, controlar melhor os níveis de estoque, reduzindo custos e oferecendo um melhor nível de serviço aos clientes. 

Conforme mostra \cite{MakridakisHibon2000}, a literatura apresenta uma grande variedade de metodologias que podem ser utilizadas para previões de demanda, sendo que a performance dos distintos modelos de previsão varia de acordo a natureza dos dados e um modelo que gera bons resultados para determinada classe de itens de uma empresa pode gerar previsões ruins para outros itens dessa mesma empresa.

Uma estratégia natural utilizada para identificar o modelo de previsão adequado pra cada item consiste em comparar a performance dos distintos modelos candidatos utilizando dados históricos de vendas do item \citep{UlrichEtAl2022}. Contudo, como geralmente as empresas produzem e/ou comercializam uma grande variedade de itens, essa estratégia acaba se tornando um esforço considerável. 

De acordo com \cite{UlrichEtAl2022}, uma opção viável consiste em agrupar os itens de acordo com seus padrões de demanda, para posteriormente identificar o modelo de previsão adequado pra cada grupo e não mais para cada item individual. O projeto elaborado por \cite{Fraga2019} propõem uma abordagem similar, contudo busca a identificação de padrões de demanda para os principais itens produzidos e/ou comercializados por um grupo de micro e pequenas empresas do Agreste Pernambucano, visando o desenvolvimento de uma metodologia combinada que seja adequada a um cojunto de padrões de demanda distintos e recorrentes nestas empresas. Como parte do projeto proposto por \cite{Fraga2019}, o GAMOS estará desenvolvendo subprojetos de pesquisa, buscando identitificar os padrões de demanda dos principais produtos de empresas de diferentes setores da região. Cada subprojeto buscará o atingimento dos seguintes objetivos:

\begin{itemize}
  \item identificação de metodologia atualmente aplicada para previsão de demanda na empresa;
  \item identificação dos principais produtos; 
  \item levantamento de dados (históricos de vendas e outros dados relevantes);
  \item compreensão de metodologias aplicadas para identificação de padrão de demanda; e
  \item análise dos padrões de demanda.
\end{itemize}
  
Os dados coletados nestes projetos, assim como os estudos realizados serão de grande importância científica, tendo em vista que serão utilizados para o desenvolvimento de uma nova metodologia de previsão de demanda combinada, e também poderão ser utilizados para outros trabalhos futuros relacionados aos setores estudados, tornando se referência para diversos estudos que venham a ser desenvolvidos.

\section{Fundamentação Teórica}

'Os padrões de demanda são resultados da variação da demanda com o tempo, ou seja, do crescimento ou declínio de taxas de demanda, sazonalidades e flutuações gerais causadas por diversos fatores' (\cite{Ballou2001} apud \cite{WernerEtAl2006}). 

De acordo com \cite{Ballou2006}, quando a demanda apresenta comportamento regular, os padrões de demanda podem ser divididos em compo­nentes de tendência, sazonais ou aleatórios. Já nos casos em que a demanda de determinados itens é inter­mitente, em função do baixo volume geral e da incerte­za quanto a quando e em que nível essa demanda ocor­rerá, a série de tempo é chamada de incerta, ou irregu­lar.

\cite{BoylanEtAl2008} distribuem os padrões de demanda entre normais, onde a demanda pode ser representada por uma distribuição normal, e não normais, no caso em que isso não é possível. De acordo com os autores, os padrões de demanda não normais podem ser classificados da seguinte forma:

\begin{itemize}
  \item um item de \emph{demanda intermitente (intermittent)} é um item com ocorrências de demanda pouco frequêntes;
  \item um item de \emph{demanda de movimento lento (slow moving)} é um item cuja demanda média por período é baixa. Isso pode ser devido a ocorrências de demanda pouco frequentes, tamanhos médios de demanda baixos ou ambos;
  \item um item de \emph{demanda errática (erratic)} é um item cujo tamanho de demanda é altamente variável;
  \item um item de \emph{demanda esporádica (lumpy)} é um item intermitente para o qual a demanda, quando ocorre, é altamente variável.; e
  \item um item de \emph{demanda agregada (clumped)} é um item intermitente para o qual a demanda, quando ocorre, é constante (ou quase constante).
\end{itemize}

Apesar da importância da identificação dos padrões de demanda para identificação dos métodos adequados de previsão, poucos autores tratam deste assunto e poucas técnicas são apresentadas na literatura para essa finalidade.

\cite{BusingerRead1999} aplicam um sistema de agrupamento de itens utilizando um diagrama de plotagem em estrela considerando oito características dos dados de séries temporais: coeficiente de variação, número de zeros, tendência, picos (outliers), sazonalidade, corridas, assimetria e autocorrelação.

O coeficiente de variação $(CV)$ informa a variabilidade em relação à média. Essa medida adimensional informa o nível de dispersão: quanto mais alto for o $(CV)$, mais alta é a dispersão:

\begin{equation}
CV = \frac{s}{\bar{i}}
\end{equation}

onde: \\

$\bar{i}$ é a média dos valores considerados, e

$s$ é o desvio padrão \\

\begin{equation}
s = \sqrt{ \frac{1}{N} \sum_{i=1}^{N}{(i-\bar{i})^2}}
\end{equation}

O número de zeros $(NZ)$ é uma medida que indica o número de períodos com demanda zero (nula) dentro de determinado intervalo de tempo. 

\begin{equation}
NZ = \sum_{i=1}^{N}{I(y_i=0)}
\end{equation}

Onde $I(A)=1$, se $A$ é verdadeiro, e $I(A)=0$, se $A$ é falso.

Essa medida é, em alguns casos, associada à intermitência. Sendo $BP$ um valor de corte, se $(NZ\geq BP)$ então a demanda é considerada intermitente \citep{BoylanEtAl2008}.

A tendência $(T)$ apresenta um padrão de variação suave e temporário na demanda. Para cáculo da tendência, \cite{BusingerRead1999} dividem os dados avaliados em terços e calculam a tendência usando a seguinte expressão:

\begin{equation}
T = \frac{(Y_U - Y_L)}{(Y_{(\frac{5}{6})} - Y_{(\frac{1}{6})})}
\end{equation}

Onde $Y_U$ e $Y_L$ representam as medianas dos terços extremos, sendo $L$ o terço inferior, e $U$ o terço superior. Observe que $-1 \geq T \leq 1$.

Picos $(P)$ é uma característica que informa sobre padrões nos dados temporais que podem representar anomalias ou dados súbitos.

\begin{equation}
P = \sum_{i=1}^{N}{I(d_i > 2)}
\end{equation}

onde:

\begin{equation}
d_i = \frac{y_i - y_{T}}{s_{T}}
\end{equation}

sendo $y_{T}$ e $s_{T}$, respectivamente, a média e o desvio padrão aparados (i.e., após retirar 20\% dos dados da amostra, sendo 10\% referente aos menores valores e 10\% referente aos maiores valores).

A sazonalidade informa comportamentos que se repetem a cada ciclo (normalmente de 4 meses). \cite{BusingerRead1999} represetam a sazonalidade através da seguinte medida adimensional:

\begin{equation}
SS = 1 - \frac{ss_{w}}{ss_{T}}
\end{equation}

onde:

\begin{equation}
ss_{w} = \sum_{i=1}^{4}\sum_{j=i}^{n_i}{x_{ij}}
\end{equation}

e

\begin{equation}
ss_{T} = \sum_{i}\sum_{j}{(x_{ij}-\bar{x})^2}
\end{equation}

sendo:

\begin{equation}
\bar{x_i} = \frac{1}{n_i} \sum_{j=1}^{n_i}{x_{ij}}
\end{equation}

e

\begin{equation}
\bar{x} = \frac{1}{n} \sum_{i=1}^{4}{n_i \bar{x_i}}
\end{equation}

tal que: \\

$y_{i}^w$ representa os dados winsorizados (i.e. após retirar 20\% dos dados da amostra, sendo 10\% referente aos menores valores e 10\% referente aos maiores valores, substituindo esses dados, respectivamente, pelo menor e pelo maior valor dentro do intervalo restante (\%80 dos dados)); \\

$n=4k+r$ representa o número total de períodos para $r = 0, 1, 2, 3$, sendo o período $(n_i)$, é definido por:

\begin{equation}
n_i = 
\begin{cases}
4k, \ \quad \mathrm{se} \quad r =0 \\
4k + i, \quad \mathrm{se} \quad r > 0
\end{cases}
\end{equation}

\begin{equation}
x_{ij} = y_{i+4(j-1)}^w \quad i = 1,...,4 \quad j=1,...,k
\end{equation}

e

\begin{equation}
x_{i(k+1)} = y_{n_i}^w \quad i = 1,...,r \quad r>0
\end{equation}

Após calculadas, as características coeficiente de variação, número de zeros, tendência, picos (outliers), sazonalidade, corridas, assimetria e autocorrelação são utilizadas por \cite{BusingerRead1999} para agurpamento dos itens, sendo então identificado qual modelo ARIMA é mais adequado para cada grupo.

\cite{Williams1984} apresenta um esquema para classificação da demanda em suave (smooth), de movimento lento, ou esporádica, particionando a variabilidade da demanda durante um lead time $(C_{LTD}^{2})$ em suas partes causais constituintes: variabilidade dos números de pedidos $(\frac{C_{n}^{2}}{\bar{L}})$, variabilidade dos tamanhos dos pedidos $(\frac{C_{x}^{2}}{\bar{n}\bar{L}})$ e variabilidade dos prazos de entrega $(C_{L}^{2})$. 

\begin{equation}
C_{LTD}^{2} = \frac{C_{n}^{2}}{\bar{L}} + \frac{C_{x}^{2}}{\bar{n}\bar{L}} + C_{L}^{2}
\end{equation}

onde: \\

$n$ representa os números de pedidos que chegam em unidades de tempo sucessivas (variáveis randômicas independetes e identicamente distribuídas (IIDRVs), com média $\bar{n}$ e variância $var(n)$), \\

$x$ representa os tamanhos dos pedidos (IIDRVs, com média $\bar{x}$ e variância $var(x)$),  \\

$L$ representa os prazos de entrega (IIDRVs, com média $\bar{L}$ e variância $var(L)$)), e \\

$C_{i}$ representa o coeficiente de variação de $i$. \\

Tal esquema foi posteriomente revisado por \cite{EavesKingsman2004}, considerando também o padrão de demanda irregular. A classificação adaptada por pelos autores é apresentada na tabela a seguir.

\begin{table}[h]
\begin{center}
\begin{tabular}[c]{c c c c}
\cline {1-4}
$\frac{C_{n}^{2}}{\bar{L}}$ & $\frac{C_{x}^{2}}{\bar{n}\bar{L}}$ & $C_{L}^{2}$ & padrão de demanda \\ \cline {1-4}
baixo & baixo &  & suave   \\ 
baixo & alto  &  & irregular   \\ 
alto  & baixo &  & de movimento lento   \\
alto  & alto  & baixo & intermitente   \\
alto  & alto  & alto  & atamente intermitente\\ \cline {1-4}
\end{tabular}
\label{tab:DemandPattern}
\caption{Classificação dos padrões de demanda de acordo com \cite{EavesKingsman2004}.}
\end{center}
\end{table}

Observe que os valores dos critérios de corte definidos para diferenciar alto e baixo são arbitrários.

\cite{SyntetosEtAl2005} sugerem um esquema de categorização contruído a partir da comparação do erro médio quadrado de três diferentes metodologias (método de Croston, método de Croton modificado e amortecimento exponecial simples). De acordo com esse esquema, os parâmetros quadrado do coeficiente de variação do tamanho da demanda $(CV^2)$ e intervalo médio entre demandas $(p)$ são usados para classificar a demanda entre errática, esporádica, suave e intermitente. A tabela a seguir apresenta a classificação proposta pelos autores.

\begin{table}[h]
\begin{center}
\begin{tabular}[c]{c c c}
\cline {1-3}
$CV^2$ & $p$ & \multirow{2}{*}{padrão de demanda} \\ 
$0.49$ & $1.32$ & \\ \cline {1-3}
baixo & baixo & suave   \\ 
baixo & alto  & intermitente   \\ 
alto  & baixo & errática   \\
alto  & alto  & esporádica  \\ \cline {1-3}
\end{tabular}
\label{tab:DemandPatternSybtetos}
\caption{Classificação dos padrões de demanda de acordo com \cite{SyntetosEtAl2005}.}
\end{center}
\end{table}

Observe que nesse esquema de classificação são definidos os pontos de corte $CV^2=0.49$ e $p=1.32$.

\section{Metodologia}

Estes projetos serão desenvolvidos através dos 5 atividades, conforme descrito a seguir:\\

Atividade 1: identificação da metodologia de previsão de demanda aplicada nas empresas - através de conversas com funcionários das empresas.\\

Atividade 2: identificação dos principais produtos - através da metodologia de classificação ABC (utilizando dados históricos dos últimos meses para todos os produtos).\\

Atividade 3: levantamento de dados históricos de vendas e dos últimos quatro anos (quando disponíveis) e de outras informações necessárias para os princiapis produtos (de acordo com classificação ABC).\\

Atividade 4: compreensão das metodologias aplicadas para identificação de padrões de demanda, conforme descritas na segunda seção.\\

Atividade 5: aplicação dos metodologias de identificação de padrões de demanda e análise dos resultados.\\

Atividade 6: preparação de relatórios e artigos para o CONIC.

\section{Resulados esperados}

Após a aplicação da metodologia acima descrita, esperamos esperamos obter os seguintes resultados: 

\begin{itemize}
    \item dados de históricos de vendas dos principais produtos das empresas estudadas;
    \item mapeamento dos padrões de demanda destes produtos;
    \item aprofundamento do conhecimento dos participantes dos projetos sobre os comportamentos de vendas e sobre a identificação de padrões de demanda;
    \item artigo apresentado no CONIC.
\end{itemize}

Os resultados destes trabalhos também serão incluídos em pelo menos um artigo que será submetido para revista.

\section{Viabilidade de execução}

Os projetos serão realizados preferencialmente no CAA-UFPE. Assim, todos os participantes destes projetos, terão acesso garantido a toda a infraestrutura necessária para o correto desenvolvimento de seu trabalho incluindo recursos físicos (sala e mobiliário), bibliográficos e computacionais da UFPE e mais especificamente do CAA. O departamento de engenharia de produção do CAA conta atualmente com dois laboratórios de informática que disponibilizam, pelo menos, 30 computadores. O GAMOS, em especial, conta com laboratório próprio, e que atualmente dispõe de 3 computadores. Em termos de recursos bibliográficos, os pesquisadores da área de engenharia da produção contam com a biblioteca central da UFPE e as bibliotecas setoriais do CTG (Centro de Tecnologia e Geociências) e do CCEN (Centro de Ciências Exatas e da Natureza), localizadas no campus da UFPE de Recife, e com a biblioteca do próprio CAA, que possuem assinatura de alguns dos principais periódicos na área além do acesso remoto à base de dados disponíveis hoje via rede entre elas o banco de dados disponibilizados pela CAPES e pelo sciencedirect. 

Parte do projeto será também realizada através de visitas às empresas que serão escolhidas para realização desse trabalho. As parcerias com as empresas serão firmadas durante a realização do projeto e os termos de parceria serão anexados no relatório final.

\section{Cronograma de Atividades}

Cada projeto estará planejado para ser realizado durante o período de 1 ano, com início previsto para 2022/2023. As atividades descritas na metodologia estão projetadas para serem realizadas conforme cronograma apresentado a seguir:

\begin{table}[h]
\begin{center}
\begin{tabular}[c]{||c||c|c|c|c|c|c||}
\cline {1-7}
\multirow{2}{*}{Atividade} & \multicolumn{6}{c ||}{Cronograma (bimestre)} \\ \cline {2-7}
 & $1^o$ & $2^o$ & $3^o$ & $4^o$ & $5^o$ & $6^o$ \\ \cline {1-7}
$1^a$ & xx &  &  &  &  &  \\
$2^a$ & xx &  &  &  &  &  \\
$3^a$ &  & xx & xx &  &  &  \\
$4^a$ &  &  & xx & xx & xx &   \\
$5^a$ &  &  & xx & xx & xx & xx  \\
$6^a$ &  &  &  &  &  & xx  \\ \cline {1-7}
\end{tabular}
\label{tab:Cronograma}
\caption{Cronograma planejado para os projetos.}
\end{center}
\end{table}


\begin{thebibliography}{9}

\bibitem[\protect\citeauthoryear{Ballou}{2001}]{Ballou2001}
Ballou, R.H. (2001).{\it Gerenciamento da Cadeia de Suprimentos: Planejamento, Organização e Logística Empresarial}, 4. ed., Porto Alegre: Bookman.

\bibitem[\protect\citeauthoryear{Ballou}{2006}]{Ballou2006}
Ballou, R.H. (2006).{\it Gerenciamento da Cadeia de Suprimentos / Logística Empresarial}, 5. ed., Porto Alegre: Bookman.

\bibitem[\protect\citeauthoryear{Boylan et al.}{2008}]{BoylanEtAl2008}
Boylan, J.E., Syntetos, A.A., e Karakostas, G.C. (2008). 'Classification for forecasting and stock control:a case study'. {\it Journal of the Operational Research Society}, Vol. 59, pp. 473--481.

\bibitem[\protect\citeauthoryear{Businger e Read}{1999}]{BusingerRead1999}
Businger, M.P., e Read, R.R. (1999). 'Identification of demand patterns for selective processing: acase study'. {\it Omega, Int. J. Mgmt Sci.}, Vol. 27, pp. 189--200.

\bibitem[\protect\citeauthoryear{Eaves e Kingsman}{2004}]{EavesKingsman2004}
Eaves A.H.C., e Kingsman B.G. (2004). 'Forecasting for the ordering and stock-holding of spare parts'. {\it J. O. Opl. Res. Soc.}, Vol. 55, pp. 431--437.

\bibitem[\protect\citeauthoryear{Fraga}{2019}]{Fraga2019}
Fraga, T.B. (2019). 'Estudo de Métodos de Previsão de Demanda e Proposição de Metodologia Combinada no Contexto das Micro e Pequenas
Empresas do Agreste Pernambucano'. Projeto de Pesquisa registrado em 09/11/2019, e aprovado pela Pró-reitoria de Pesquisa da UFPE em 11/02/2021 (Processo SIPAC 23076.057489/2019-21).

\bibitem[\protect\citeauthoryear{Makridakis et al.}{1998}]{MakridakisEtAl1998}
Makridakis, S.G.,Wheelwright, S.C., Hyndman, R.J. (1998).{\it Forecasting: Methods and Applications}, 3. ed., Wiley.

\bibitem[\protect\citeauthoryear{Makridakis e Hibon}{2000}]{MakridakisHibon2000}
Makridakis, S. e Hibon, M. (2000) 'The M3-Competition: results, conclusions and implications'. {\it International Journal of Forecasting}, Vol. 16, pp. 451--476.

\bibitem[\protect\citeauthoryear{Syntetos et al.}{2005}]{SyntetosEtAl2005}
Syntetos, A.A., Boylan, J.E., e Croston, J.D. (2005) 'On the categorization of demand patterns'. {\it Journal of the Operational Research Society}, Vol. 56 (5), pp. 495--503.

\bibitem[\protect\citeauthoryear{Ulrich et al.}{2022}]{UlrichEtAl2022}
Ulrich, M., Jahnke, H., Langrock, R., Pesch, R., e Senge, R. (2022) 'Classification-based model selection in retail demand forecasting'. {\it International Journal of Forecasting}, Vol. 38 (1), pp. 209--223.

\bibitem[\protect\citeauthoryear{Werner et al.}{2006}]{WernerEtAl2006}
Werner, L, Lemos, F.O., Daudt, T. (2006) 'Previsão de demanda e níveis de estoque uma abordagem conjunta aplicada no setor siderúrgico'. {\it XIII SIMPEP}, Bauru, SP, Brasil.

\bibitem[\protect\citeauthoryear{Williams}{1984}]{Williams1984}
Williams, T.M. (1984). 'Stock control with sporadic and slow-moving demand'. {\it Journal of the Operational Research Society}, Vol. 35 (10), pp. 939–948. 

\end{thebibliography} 

\end{document}

