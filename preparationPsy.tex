\documentclass{book}
\usepackage[brazilian]{babel}
\usepackage[utf8]{inputenc}
\usepackage[T1]{fontenc}

\title{Preparação para a conversa com o psiquiatra}

\author{Tatiana Balbi Fraga$^{1*}$}

\begin{document}

\emph{09/02/2022}  \\

No dia 14 de janeiro de 2022, eu fui internada em uma clínica. Voltei para casa no dia 07 de fevereiro de 2022. Eu fui internada contra a minha vontade. Eu estava em meu apartamento, na praia de Carneiros, fui levada até a clínica em Recife de carro por pessoas que eu não conheço. \\

Chegando na clínica, eu saí do carro, fui conduzida até uma psiquiatra, conversei com elas por algum tempo, depois fui conduzida até dentro da clínica. Eu fui conduzida, e segui pacificamente, simplesmente porque eu não tinha idéia que eu seria internada. \\

Eu voltei para minha casa, em 07 de fevereiro de 2022. \\

Quando cheguei em casa, encontrei meu gatil completamente abandonado. Muito mato em todo o terreno. Inclusive dentro do gatil. A porta do gatil estava aberta, muitas coisas entulhadas dentro do quarto dos gatinhos. O banheiro do gatil muito sujo. Tudo muito bagunçado... e os gatinhos assustados. Foi muito triste constatar que minha mãe me internou em uma clínica... não apenas contra a minha vontade... mas sem que eu sequer soubesse o que estava acontecendo... não permitiu que eu cuidasse dos meus gatinhos, me internando em uma clínica... e ainda abandonou meus gatinhos em minha casa... minha mãe foi para a praia. \\

Ela me internou em uma clinica, me impedindo de zelar pelos meus gatinhos, abandonou meus gatinhos em minha casa... e foi para a praia.\\

Depois de todo esse inferno. Me faltam três gatinhos. Não sei se estão vivos ou mortos. \\

Tanta coisa eu perdi nesses últimos dias, que mal posso contar. Mal posso acreditar que isso tudo esteja acontecendo comigo. \\

Quando eu sai da clínica, foi sob um acordo com um psicólogo, eu iria ter seções com o psicólogo e com um psiquiatra.\\

Mal posso acreditar em tudo isso... \\

Agora tenho uma reunião com um psiquiatra, e preciso organizar minha retórica, para não ser taxada de louca. \\

Sobre o que pretendo falar com o psiquiatra: \\

1) não tenho absolutamente nenhum problema psicológico: \\

\noindent - Não tenho Transtorno obsessivo conpulsivo; \\
- Não tenho transtorno bipolar; \\
- Não sou esquisofrênica; \\
- Não tenho prolemas psicóticos; \\
- resumindo... não tenho absolutamente nenhum problema mental que justifique que eu siga com medicações. \\

2) eu não tenho a necessidade de ter acompanhamento psiquiátrico. \\

3) eu não sou dependente de minha mãe: \\

\noindent - minha mãe sempre foi uma mãe relapsa; \\
- quando eu era criança, ela passava a semana trabalhando e final de semana ia se encontrar com namorado; \\
- eu nunca tive uma mãe presente; \\
- durante toda a minha infancia, eu nunca tive o carinho materno; \\
- em meu aniversário, sempre havia uma festa, mas eu não ganhava presentes; \\
- ela sempre disse que criava os filhos para o mundo. \\

\noindent - isso tudo me fez ser independete e hoje eu sou uma pessoa independente; \\

\noindent - eu trabalho; \\
- pago as minhas contas; \\
- eu sou uma pessoa focada e esfoçada; \\
- sempre fiz um grande esfoço para construir as coisas em minha vida; \\
- sou uma pessoa boa; \\
- sou caridosa (dedico minha vida para cuidar de gatinhos que eu encontrei na rua); \\
- novamente vou atestar... eu não tenho absolutamente nenhum problema mental. \\

Revivendo os fatos anteriores à minha internação na clínica... \\




\end{document}

