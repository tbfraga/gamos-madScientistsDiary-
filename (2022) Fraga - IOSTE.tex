\documentclass{book}
\usepackage[brazilian]{babel}
\usepackage[utf8]{inputenc}
\usepackage[T1]{fontenc}

\begin{document}

\emph{Practice shows the way} \\

In 2013 I started teaching Mathematical Programming. As most Operations Research books deal with Mathematical Programming, the coordinator of the Production Engineering Course (in which I was assigned), recommended that I use Operations Research books for the discipline. And, as I didn't understand well what Mathematical Programming was about, I ended up mixing the two sciences. At the end of the semester, the students were completely dissatisfied for having studied the same subject in two disciplines (Operations Research and Mathematical Programming). \\

The students' dissatisfaction motivated me to research more deeply on the subject. That's when I found the book. Through this book I was able to understand that: \\

Mathematical Programming is a science that studies the methods and concepts applied to the modeling and 'mathematical solution' of Mathematical Programming Problems. And not least, I realized that Mathematical Programming Problems are basically Optimization Problems. That is, problems in which one seeks, among a known number of possible solutions (finite or infinite), to find a viable solution that best meets the objective or set of objectives considered. Where a viable solution is defined as a solution that satisfies the set of constraints defined by the problem. \\

In the second semester in which I taught the course, I had the happy idea of passing a work to the students, through which they should search, understand and present scientific articles with the presentation of mathematical models representing real optimization problems. \\

\end{document}

