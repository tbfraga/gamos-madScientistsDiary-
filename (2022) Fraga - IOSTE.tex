\documentclass{book}
\usepackage[brazilian]{babel}
\usepackage[utf8]{inputenc}
\usepackage[T1]{fontenc}

\begin{document}

\emph{Practice shows the way} \\

In 2013 I started teaching Mathematical Programming. As most of the Operations Research books deal with Mathematical Programming, the coordinator of the Production Engineering Course recommended that I use Operations Research books for the discipline. And, as I didn't understand well what Mathematical Programming was about, I ended up mixing the two sciences. At the end of the semester, the students were completely dissatisfied for having studied the same subject in two disciplines (Operations Research and Mathematical Programming). \\

The students' dissatisfaction motivated me to research more deeply on the subject. That's when I found the Mathematical Programming journal of Springer (https://www.springer.com/journal/10107). Through this journal I was able to understand that: \\

\begin{quotation}
\noindent Mathematical Programming is a science that studies the methods and concepts applied to the modeling and mathematical solution of Mathematical Programming Problems. \\
\end{quotation}

And not least, I realized that: \\

\begin{quotation}

\noindent Mathematical Programming Problems are basically Optimization Problems. That is, problems in which one seeks, among a known number of possible solutions (finite or infinite), to find a viable solution that best meets the objective or set of objectives considered. Where a viable solution is defined as a solution that satisfies the set of constraints defined by the problem. \\

\end{quotation}

In the second semester in which I taught the course, I had the happy idea of passing a work to the students, through which they should search, understand and present scientific articles with the presentation of mathematical models representing real optimization problems. And, I was incredibly pleased to realize that, despite the great difficulty of performing this task, the students loved doing the work. \\

Over the next few semesters, I slowly began to fall in love with math and teaching math. And I started to understand math as a language, just like English, French or Portuguese. But with the incredible advantage of being a perfect language, which leaves no room for misinterpretation. \\

Through my former companion, I discovered that Catholic monks, when beginning to understand Arabic mathematical books, deduced that mathematics was a divine language. And I concluded, that such perfection could only come from God. When then I started to develop new works with the students: real problem modeling works. \\

So, I started to transform my Mathematics Programming classes into directed studies. Two tactics were applied. The first practice involved working in the classroom teaching the mathematical language and methods through the modeling and solving of some cases from the book by Hillier and Lieberman \cite{HillierLieberman}. The second practice involved asking students to identify optimization problems around them and model those problems. \\

Some students really got involved with the discipline, and we had some interesting papers published at conferences, with emphasis on the work of locating water filters at the university \cite{AndradeAndFraga2016}. \\

It was then that I realized that Mathematics Programming was not enough. Motivated by the resolution of UFPE (the university where I work), I understood that I could go further with the students, uniting teaching, research and extension. And based on that enlightenment, already in early 2014 I designed the course 'Special Topics in Production Engineering 2', through which I started to show students, in an increasingly complex way, the modeling of Standard Problems of Combinatorial Optimization, starting with Assignment Problems, going through Routing Problems and going to the famous Machine Scheduling Problems. Being of great importance, I also presented to the students a few methods and heuristics used to solve such problems. \\

At the same time, I started to develop projects with the students through which real combinatorial optimization problems, usually from companies, were mathematically modeled, and solved using commercial software such as LINGO, or even through the development of computer codes. Among such projects, extension projects, research, scientific initiation projects, or student monographs stood out. Such works were developed with the following dynamics. Students carry out an in-depth study of some process, describing it in detail. Within the process described by the student, I identified some combinatorial optimization problem, I modeled the problem mathematically, explaining the model to the students in detail, and usually I created some solver to solve the problem, also explaining to the student in detail the methodology behind of the solver. Some students were motivated to participate in the development of computer codes to solve problems. \\

Some of the diverse results of these combined efforts have resulted in interesting publications \\

It was incredible to perceive in each student the happiness in understanding and being able to explain the processes and problems with such perfection. As well as each equation of the mathematical models developed and the algorithms applied for solution. In some cases, I had the great happiness of introducing my students to the art of programming. \\

In addition to the modeling work, other applied work (analysis and improvement of processes in companies) were also developed with the students \cite{AraujoEtAl2020}. \\

Again, this joint advance made me check a new limitation: programming. Combinatorial Optimization and Programming are two highly integrated sciences. Since the first cannot walk without the second. Mathematical modeling is amazing in the sense that it allows a detailed understanding of a real problem, every nuance. However, once a model is built, for which it finds the desired answer to the problem, or at least a somewhat close answer, programming simply becomes indispensable. \\

Currently I understand that Mathematical Modeling and Programming must go hand in hand, since one shapes the other. \\

This new lighting made me launch in 2017 the extension project 'Applied Programming Course'. Through this course I taught students: Object Oriented Programming Practices (first module); and Elaboration and development of projects focused on the identification, modeling and development of algorithms for solving Combinatorial Optimization Problems (second module). \\

The Applied Programming Course also resulted in some important works.

The results of all these actions were incredible. Not only because of the work developed and the scientific contributions achieved, but mainly because it has awakened a different look at mathematics and programming from the various students involved. I believe that by delving deeply into the world of mathematics we become a little crazy, a little like Albert Einstein. We develop reasoning primarily based on logic. \\

Such a development of consciousness contributes to the students having new horizons, new job opportunities. Especially when we are faced with the expansion of industry 4.0. \\

Perhaps, a necessary new step (limitation) is the expansion of these works, focusing on the issue of relationships. I believe that mathematicians/programmers isolate themselves a bit from society, and emotional and social health training becomes necessary. Since the university (like any organization) is a dynamic and social organism, relationships become a basic necessity for the healthy development of any project, and I hope that this will also be the focus of my future work. \\

Unfortunately, currently I am no longer teaching subjects that directly involve mathematics, however I am working to finalize and publish in English all the works developed together with my advisees. I hope that soon we will be able to finalize and make available to everyone new motivations to awaken the interest of many in this incredible science - mathematics. 


\bibliographystyle{plain}
\bibliography{bibfile}

\end{document}

