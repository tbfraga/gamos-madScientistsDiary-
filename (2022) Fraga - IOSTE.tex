\documentclass{book}
\usepackage[brazilian]{babel}
\usepackage[utf8]{inputenc}
\usepackage[T1]{fontenc}

\begin{document}

\emph{Practice shows the way} \\

In 2013 I started teaching Mathematical Programming. As most Operations Research books deal with Mathematical Programming, the coordinator of the Production Engineering Course (in which I was assigned), recommended that I use Operations Research books for the discipline. And, as I didn't understand well what Mathematical Programming was about, I ended up mixing the two sciences. At the end of the semester, the students were completely dissatisfied for having studied the same subject in two disciplines (Operations Research and Mathematical Programming). \\

The students' dissatisfaction motivated me to research more deeply on the subject. That's when I found the book. Through this book I was able to understand that: \\

Mathematical Programming is a science that studies the methods and concepts applied to the modeling and 'mathematical solution' of Mathematical Programming Problems. And not least, I realized that Mathematical Programming Problems are basically Optimization Problems. That is, problems in which one seeks, among a known number of possible solutions (finite or infinite), to find a viable solution that best meets the objective or set of objectives considered. Where a viable solution is defined as a solution that satisfies the set of constraints defined by the problem. \\

In the second semester in which I taught the course, I had the happy idea of passing a work to the students, through which they should search, understand and present scientific articles with the presentation of mathematical models representing real optimization problems. And, I was incredibly pleased to realize that, despite the great difficulty of performing this task, the students loved doing the work. \\

Over the next few semesters, I slowly began to fall in love with math and teaching math. And I started to understand math as a language, just like English, French or Portuguese. But with the incredible advantage of being a perfect language, which leaves no room for misinterpretation. \\

Through my former companion, I discovered that Catholic monks, when beginning to understand Arabic mathematical books, deduced that mathematics was a divine language. And I concluded, that such perfection could only come from God. When then I started to develop new works with the students: real problem modeling works. \\

So, I started to transform my Mathematics Programming classes into directed studies. Two tactics were applied. The first practice involved working in the classroom teaching the mathematical language through the modeling of book problems. The second practice involved asking students to identify optimization problems around them and model those problems. \\

Some students really got involved with the discipline, and I had some interesting papers published at conferences with students. \\

It was then that I realized that Mathematics Programming was not enough. Motivated by the resolution of UFPE (the university where I work), I understood that I could go further with the students, uniting teaching, research and extension. And based on that enlightenment, already in early 2014 I designed the course 'Special Topics in Production Engineering 2', through which I started to show students, in an increasingly complex way, the modeling of Standard Problems of Combinatorial Optimization , starting with Assignment Problems, going through Routing Problems and going to the famous Machine Scheduling Problems. Being of great importance, I also presented to the students a few methods and heuristics used to solve such problems. \\

At the same time, I started to develop projects with the students through which real combinatorial optimization problems, usually in companies, were mathematically modeled, and solved using commercial software such as LINGO, or even through the development of computer codes. Among such projects, extension projects, research, scientific initiation projects, or student monographs stood out. Such works were developed with the following dynamics. Students carry out an in-depth study of some process, describing it in detail. Within the process described by the student, I identified some combinatorial optimization problem, I modeled the problem mathematically, explaining the model to the students in detail, and usually I created some solver to solve the problem, also explaining to the student in detail the methodology behind of the solver. \\

It was incredible to perceive in each student the happiness in understanding and being able to explain the processes and problems with such perfection. As well as each equation of the mathematical models developed and the algorithms applied for solution. In some cases, I had the great happiness of introducing my students to the art of programming. \\

Again, this joint advance made me check a new limitation: programming. Combinatorial Optimization and Programming are two highly integrated sciences. Since the first cannot walk without the second. Mathematical modeling is amazing in the sense that it allows a detailed understanding of a real problem, every nuance. However, once a model is built, for which it finds the desired answer to the problem, or at least a somewhat close answer, programming simply becomes indispensable. \\





\end{document}

