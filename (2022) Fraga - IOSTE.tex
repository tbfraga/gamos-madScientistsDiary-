\documentclass{book}
\usepackage[brazilian]{babel}
\usepackage[utf8]{inputenc}
\usepackage[T1]{fontenc}

\begin{document}

\emph{Practice shows the way} \\

In 2013 I started teaching Mathematical Programming. As most Operations Research books deal with Mathematical Programming, the coordinator of the Production Engineering Course (in which I was assigned), recommended that I use Operations Research books for the discipline. And, as I didn't understand well what Mathematical Programming was about, I ended up mixing the two sciences. At the end of the semester, the students were completely dissatisfied for having studied the same subject in two disciplines (Operations Research and Mathematical Programming). \\

The students' dissatisfaction motivated me to research more deeply on the subject. That's when I found the book. Through this book I was able to understand that: \\

Mathematical Programming is a science that studies the methods and concepts applied to the modeling and 'mathematical solution' of Mathematical Programming Problems. And not least, I realized that Mathematical Programming Problems are basically Optimization Problems. That is, problems in which one seeks, among a known number of possible solutions (finite or infinite), to find a viable solution that best meets the objective or set of objectives considered. Where a viable solution is defined as a solution that satisfies the set of constraints defined by the problem. \\

In the second semester in which I taught the course, I had the happy idea of passing a work to the students, through which they should search, understand and present scientific articles with the presentation of mathematical models representing real optimization problems. And, I was incredibly pleased to realize that, despite the great difficulty of performing this task, the students loved doing the work. \\

Over the next few semesters, I slowly began to fall in love with math and teaching math. And I started to understand math as a language, just like English, French or Portuguese. But with the incredible advantage of being a perfect language, which leaves no room for misinterpretation. \\

Through my former companion, I discovered that Catholic monks, when beginning to understand Arabic mathematical books, deduced that mathematics was a divine language. And I concluded, that such perfection could only come from God. When then I started to develop new works with the students: real problem modeling works. \\

So, I started to transform my Mathematics Programming classes into directed studies. Two tactics were applied. The first practice involved working in the classroom teaching the mathematical language through the modeling of book problems. The second practice involved asking students to identify optimization problems around them and model those problems. \\





\end{document}

