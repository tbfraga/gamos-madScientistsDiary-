\documentclass[11pt,latterpaper,twocolumn]{article}
\usepackage[english]{babel}
\usepackage[utf8]{inputenc}
\usepackage[T1]{fontenc}
\usepackage{float}
\usepackage{xcolor}
\usepackage{verbatim}
\usepackage{charter}
\usepackage{amsmath}
%\usepackage{appendix}
\usepackage{ragged2e}
\usepackage{array}
\usepackage{etoolbox}
\usepackage{fancyhdr}
\usepackage{booktabs}
%\usepackage{arydshln}
\usepackage{caption}
\usepackage{subcaption}
%\usepackage{enumitem}
\usepackage{geometry}
\geometry{
  top=0.8in,            
  inner=0.5in,
  outer=0.5in,
  bottom=0.9in,
  headheight=4ex,       
  headsep=6.5ex,         
}
\usepackage{graphicx}
\usepackage{mathtools}
%\usepackage{multirow}
\usepackage{pdfpages}
%\usepackage{subfiles}
%\usepackage[compact]{titlesec}
%\usepackage{stfloats}
%\usepackage{natbib,stfloats}
\usepackage[parfill]{parskip}
%\usepackage[style=abnt]{biblatex}

\setlength{\columnsep}{30pt}

\pagestyle{fancy}
\fancyhf{}

\definecolor{blueM}{cmyk}{1.0,0.49,0.0,0.47}
      
\lfoot{\textcolor{blueM}{Fraga, T.B. (\the\year). What I'm working on now. \emph{Mad Scientists Diary}.}}
\rfoot[C]{\textcolor{blueM}{\thepage}} % page
\renewcommand{\headrulewidth}{0mm} % headrule width
\renewcommand{\footrulewidth}{0mm} % footrule width

\makeatletter
\patchcmd{\headrule}{\hrule}{\color{black}\hrule}{}{} % headrule
\patchcmd{\footrule}{\hrule}{\color{black}\hrule}{}{} % footrule
\makeatother
    
\fancypagestyle{firstpage}
{   
    \chead[]{
      \begin{tabular}{m{1.6cm}m{11.4cm}m{2.5cm}}
      \includegraphics[height=1.5cm]{images/Einstein.png} 
      &
      \centering
     \fcolorbox{white}{blueM}{\fbox{\begin{minipage}{11.5cm}
     \centering
     \textcolor{white}{Mad Scientists Diary}
     \end{minipage}}}
         &
        \centering
         \tiny{ \vspace{3.5mm} GAMOS\\%Para usar en otras carreras consulte a su coordinador
%%%%%%%%%%%%%%%%%%%%%%%%%%%%%%%%%%%%%%%%%%%%%%%%%%%%%%%%%%%%%%%%%%%%%%%%%%%%%%%%%%%%%%%%%%%%%%%%%
          publication of works in progress\\ %Elija el ciclo que corresponda
%%%%%%%%%%%%%%%%%%%%%%%%%%%%%%%%%%%%%%%%%%%%%%%%%%%%%%%%%%%%%%%%%%%%%%%%%%%%%%%
          }\tabularnewline
         %\hline
         \end{tabular}
   }
}
    
\begin{document}

\twocolumn[\begin{@twocolumnfalse}

\hspace{25pt}
\begin{minipage}{0.90\textwidth}
\thispagestyle{firstpage}
\vspace{5mm}
    \center\Large{\textbf{What I'm working on now}} 
    \vspace{3mm}
    
    \large{\textbf{Tatiana Balbi Fraga}} 
    \vspace{2mm}
    
    %\large{\textbf{Asesor 1: Nombre del Asesor$^1$} ; Asesor 2: Nombre del Asesor$^2$} \newline
    %Si solo hay un asesor borrar el ``1''
    \fontsize{0.35cm}{0.5cm}\selectfont \textit{Centro Acadêmico do Agreste, Universidade Federal de Pernambuco}
    \vspace{1mm} 
    
    \today % FECHA
\end{minipage}

\small

\vspace{11pt}

\centerline{\rule{0.95\textwidth}{0.4pt}}

\begin{center}
    
    \begin{minipage}{0.9\textwidth}
        % RESUMEN
        \noindent \textbf{Abstract:} In this paper I present a summary of the work that the GAMOS team is currently developing with my participation and coordenation. 
    
        \vspace{4mm}
        % PALABRAS CLAVE
        \noindent \textbf{Key words:} GAMOS, work in progress.
    
    \end{minipage}
    
\end{center}

\centerline{\rule{0.95\textwidth}{0.4pt}}

\vspace{15pt}
\end{@twocolumnfalse}]
%%%%%%%%%%%%%%%%%%%%%%%%%%%%%%%%%%%%%%%%%%%%%%%%%%%%%%%%%%%%

\section{GAMOS work in progress}

It's now halfway through June and I'm coordinating and participating in two research projects - 'Study of Demand Forecasting Methods and Proposition of a Combined Methodology in the Context of Micro and Small Enterprises in Agreste Pernambucano' and 'Analysis and Modeling of Continuous and Combinatorics Optimization Problems'. As explained better in the following sections.

\section{Study of Demand Forecasting Methods and Proposition of a Combined Methodology in the Context of Micro and Small Enterprises in Agreste Pernambucano}

This project was proposed on 2019/11/09 but accepted by the UFPE research board only on 2021/02/11, when we started developing it. Since I'm doing a lot of work simultaneously, I have not yet gone further with this project.

So far, as part of this project, I am supervising two scientific initiation projects: a PIBIC project entitled 'Identification of demand patterns for a company in the plastics sector' and an undergraduate project entitled ' 'Identification of demand patterns for a company in the furniture trade sector'. In these two projects we were able to carry out a large survey of data from two companies in these two sectors studied and, due to the large number of products, we used a multicriteria method for ABC classification of the products.

In this second part of the semester, we will apply methods to identify the demand pattern for products identified as class A by the multicriteria classification method used.

We hope to finalize this part of the work with a publication in a congress, and possibly in a scientific journal.

As future work, we are going to identify demand forecasting methods suited to the patterns studied and propose a combined methodology, comparing the results found from the combined methodology and the methodologies used in the combination.

\section{Analysis, Modeling and Solution of Continuous and Combinatorics Optimization Problems}

I have alread fineshed the publication of two problems I were working on - 

%\begin{thebibliography}{9}

%\bibitem[\protect\citeauthoryear{Ballou}{2001}]{Ballou2001}
%Ballou, R.H. (2001).{\it Gerenciamento da Cadeia de Suprimentos: Planejamento, Organização e Logística Empresarial}, 4. ed., Porto Alegre: Bookman.

%\bibitem[\protect\citeauthoryear{Ballou}{2006}]{Ballou2006}
%Ballou, R.H. (2006).{\it Gerenciamento da Cadeia de Suprimentos / Logística Empresarial}, 5. ed., Porto Alegre: Bookman.

%\bibitem[\protect\citeauthoryear{Boylan et al.}{2008}]{BoylanEtAl2008}
%Boylan, J.E., Syntetos, A.A., e Karakostas, G.C. (2008). 'Classification for forecasting and stock control:a case study'. {\it Journal of the Operational Research Society}, Vol. 59, pp. 473--481.

%\bibitem[\protect\citeauthoryear{Businger e Read}{1999}]{BusingerRead1999}
%Businger, M.P., e Read, R.R. (1999). 'Identification of demand patterns for selective processing: acase study'. {\it Omega, Int. J. Mgmt Sci.}, Vol. 27, pp. 189--200.

%\bibitem[\protect\citeauthoryear{Eaves e Kingsman}{2004}]{EavesKingsman2004}
%Eaves A.H.C., e Kingsman B.G. (2004). 'Forecasting for the ordering and stock-holding of spare parts'. {\it J. O. Opl. Res. Soc.}, Vol. 55, pp. 431--437.

%\bibitem[\protect\citeauthoryear{Fraga}{2019}]{Fraga2019}
%Fraga, T.B. (2019). 'Estudo de Métodos de Previsão de Demanda e Proposição de Metodologia Combinada no Contexto das Micro e Pequenas
%Empresas do Agreste Pernambucano'. Projeto de Pesquisa registrado em 09/11/2019, e aprovado pela Pró-reitoria de Pesquisa da UFPE em 11/02/2021 (Processo SIPAC 23076.057489/2019-21).

%\bibitem[\protect\citeauthoryear{Makridakis et al.}{1998}]{MakridakisEtAl1998}
%Makridakis, S.G.,Wheelwright, S.C., Hyndman, R.J. (1998).{\it Forecasting: Methods and Applications}, 3. ed., Wiley.

%\bibitem[\protect\citeauthoryear{Makridakis e Hibon}{2000}]{MakridakisHibon2000}
%Makridakis, S. e Hibon, M. (2000) 'The M3-Competition: results, conclusions and implications'. {\it International Journal of Forecasting}, Vol. 16, pp. 451--476.

%\bibitem[\protect\citeauthoryear{Syntetos et al.}{2005}]{SyntetosEtAl2005}
%Syntetos, A.A., Boylan, J.E., e Croston, J.D. (2005) 'On the categorization of demand patterns'. {\it Journal of the Operational Research Society}, Vol. 56 (5), pp. 495--503.

%\bibitem[\protect\citeauthoryear{Ulrich et al.}{2022}]{UlrichEtAl2022}
%Ulrich, M., Jahnke, H., Langrock, R., Pesch, R., e Senge, R. (2022) 'Classification-based model selection in retail demand forecasting'. {\it International Journal of Forecasting}, Vol. 38 (1), pp. 209--223.

%\bibitem[\protect\citeauthoryear{Werner et al.}{2006}]{WernerEtAl2006}
%Werner, L, Lemos, F.O., Daudt, T. (2006) 'Previsão de demanda e níveis de estoque uma abordagem conjunta aplicada no setor siderúrgico'. {\it XIII SIMPEP}, Bauru, SP, Brasil.

%\bibitem[\protect\citeauthoryear{Williams}{1984}]{Williams1984}
%Williams, T.M. (1984). 'Stock control with sporadic and slow-moving demand'. {\it Journal of the Operational Research Society}, Vol. 35 (10), pp. 939–948. 

%\end{thebibliography} 

\end{document}

