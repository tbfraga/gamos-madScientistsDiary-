\documentclass[11pt,letterpaper,twocolumn]{article}
\usepackage[brazilian, english]{babel}
\usepackage[utf8]{inputenc}
\usepackage[T1]{fontenc}
\usepackage{float}
\usepackage{xcolor}
\usepackage{verbatim}
\usepackage{charter}
\usepackage{amsmath}
\usepackage{appendix}
\usepackage{ragged2e}
\usepackage{array}
\usepackage{etoolbox}
\usepackage{fancyhdr}
\usepackage{booktabs}
\usepackage{arydshln}
\usepackage{caption}
\usepackage{subcaption}
\usepackage{enumitem}
\usepackage{geometry}
\geometry{
  top=0.8in,            
  inner=0.5in,
  outer=0.5in,
  bottom=0.9in,
  headheight=4ex,       
  headsep=6.5ex,         
}
\usepackage{graphicx}
\usepackage{mathtools}
\usepackage{multirow}
\usepackage{pdfpages}
\usepackage{subfiles}
\usepackage[compact]{titlesec}
\usepackage{stfloats}

\setlength{\columnsep}{30pt}

\pagestyle{fancy}
\fancyhf{}
      
\fancyfoot{}
\fancyfoot[C]{\thepage} % page
\renewcommand{\headrulewidth}{0mm} % headrule width
\renewcommand{\footrulewidth}{0mm} % footrule width

\makeatletter
\patchcmd{\headrule}{\hrule}{\color{black}\hrule}{}{} % headrule
\patchcmd{\footrule}{\hrule}{\color{black}\hrule}{}{} % footrule
\makeatother

\definecolor{blueM}{cmyk}{1.0,0.49,0.0,0.47}

\chead[C]{
      \begin{tabular}{m{1.6cm}m{11.4cm}m{2.5cm}}
      \includegraphics[height=1.5cm]{images/Einstein.png} 
      &
      \centering
     \fcolorbox{white}{blueM}{\fbox{\begin{minipage}{11.5cm}
     \centering
     \textcolor{white}{Mad Scientists Diary}
     \end{minipage}}}
         &
        \centering
         \tiny{ \vspace{3.5mm} GAMOS\\%Para usar en otras carreras consulte a su coordinador
%%%%%%%%%%%%%%%%%%%%%%%%%%%%%%%%%%%%%%%%%%%%%%%%%%%%%%%%%%%%%%%%%%%%%%%%%%%%%%%%%%%%%%%%%%%%%%%%%
          publication of works in progress\\ %Elija el ciclo que corresponda
%%%%%%%%%%%%%%%%%%%%%%%%%%%%%%%%%%%%%%%%%%%%%%%%%%%%%%%%%%%%%%%%%%%%%%%%%%%%%%%
          }\tabularnewline
%          \hline
          \end{tabular}%
    }
    
\begin{document}
\twocolumn[\begin{@twocolumnfalse}


%\begin{minipage}{0.15\textwidth}{
%    \includegraphics[width=4cm]{images/Einstein.png}}
%\end{minipage}
\hspace{25pt}
\begin{minipage}{0.75\textwidth}
\vspace{5mm}
    \Large{\textbf{Starting a new challenge}} 
    \vspace{3mm}
    
    \large{\textbf{Tatiana Balbi Fraga}} 
    \vspace{2mm}
    
    %\large{\textbf{Asesor 1: Nombre del Asesor$^1$} ; Asesor 2: Nombre del Asesor$^2$} \newline
    %Si solo hay un asesor borrar el ``1''
    \fontsize{0.35cm}{0.5cm}\selectfont \textit{Núcleo de Tecnologia, Centro Acadêmico do Agreste, Universidade Federal de Pernambuco}
    \vspace{1mm} 
    
    Dec 20, 2021 to \today % FECHA

\end{minipage}

\small

\vspace{11pt}

\centerline{\rule{0.95\textwidth}{0.4pt}}

\begin{center}
    
    \begin{minipage}{0.9\textwidth}
        % RESUMEN
        \noindent \textbf{Abstract:} Idea and conception of the scientific article developed for IOSTE.
    
        \vspace{4mm}
        % PALABRAS CLAVE
        \noindent \textbf{Key words:} IOSTE, practice-based learning, teaching strategy.
    
    \end{minipage}
    
\end{center}

\centerline{\rule{0.95\textwidth}{0.4pt}}

\vspace{15pt}
\end{@twocolumnfalse}]
%%%%%%%%%%%%%%%%%%%%%%%%%%%%%%%%%%%%%%%%%%%%%%%%%%%%%%%%%%%%

\emph{Dec 20, 2021} \\

Today I woke up thinking about how it would be possible to take the idea of -madScientistsDiary- forward. What would be the limitations and difficulties, how would it be possible to disseminate information about the work in progress... if that would be possible.

I was also preoccupied with issues such as newspaper formatting and everything else.

So, the possibility of a pilot project emerged, in which the practice would demonstrate the best path and the difficulties.

Just now I learned about the launch of the call for papers for the Symposium IOSTE 2022.

https://ioste.org/ioste\_brochure\_2020-2022.pdf

Given a number of factors that have converged in recent days, I decided to submit a proposal for this symposium.

At that point I understood, that dissemination can even start with an idea. Perhaps it is even recommended that this be so...

My idea for IOSTE: is it possible to unite research, collaborative work, information technology and transparency in the quest to improve undergraduate engineering education?

I will soon format this first report in the form of a newspaper article, with bibliographical references, abstract and whatever else is appropriate. I hope this will be the first article in -madScientistsDiary- and that it will bring good results and many other ideas. \\

\emph{Dec 21, 2021} \\

The funniest part of the text I present here is that within the text the theme that will be worked on in the IOSTE article emerged very explicitly. And I believe that this could be the theme: practice shows the way.

This topic is directly associated with the work I developed with students in my extension projects. I developed such projects, the lesson plans of the courses I teach, and everything else, always with this focus: teaching through practice. And, of course, this is directly related to research and extension. \\

But it's not always easy to do this... \\

\emph{May 4, 2022} \\

Finally the Mad Scientists Diary got its new format. 

On the proposal for the IOSTE symposium \cite{IOSTE}, the article "Practice shows the way" was prepared and recently approved. It presents a university teaching strategy, grounded on a practice-based learning approach, which unites research, extension and technology seeking to improve the teaching of Mathematical Modeling and Optimization.

\begin{thebibliography}{9}
\bibitem{IOSTE} IOSTE (2022). ``Esperançar in uncertain times: the role of science and technology education in a changing world.'' \textit{XX IOSTE International Symposium 2022}. Recife, Brazil. https://ioste2022.com.
\end{thebibliography} 

\end{document}

