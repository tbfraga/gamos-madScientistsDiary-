\documentclass{book}
\usepackage[brazilian]{babel}
\usepackage[utf8]{inputenc}
\usepackage[T1]{fontenc}

\title{Preparation of an article for submission to IOSTE 2022}

\author{Tatiana Balbi Fraga$^{1*}$}

\begin{document}

Text under construction \\

Today I woke up thinking about how it would be possible to take the idea of -madScientistsDiary- forward. What would be the limitations and difficulties, how would it be possible to disseminate information about the work in progress... if that would be possible.

I was also preoccupied with issues such as newspaper formatting and everything else.

So, the possibility of a pilot project emerged, in which the practice would demonstrate the best path and the difficulties.

Just now I learned about the launch of the call for papers for the Symposium IOSTE 2022.

https://ioste.org/ioste\_brochure\_2020-2022.pdf

Given a number of factors that have converged in recent days, I decided to submit a proposal for this symposium.

At that point I understood, that dissemination can even start with an idea. Perhaps it is even recommended that this be so...

My idea for IOSTE: is it possible to unite research, information technology and transparency in the quest to improve undergraduate engineering education?

I will soon format this first report in the form of a newspaper article, with bibliographical references, abstract and whatever else is appropriate. I hope this is the first article in -madScientistsDiary- and that it will bring good results and many other ideas.

\end{document}

