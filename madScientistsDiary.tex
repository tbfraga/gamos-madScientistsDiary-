\documentclass{book}
\usepackage[brazilian]{babel}
\usepackage[utf8]{inputenc}
\usepackage[T1]{fontenc}

\title{Preparation of an article for submission to IOSTE 2022}

\author{Tatiana Balbi Fraga$^{1*}$}

\begin{document}

Text under construction \\

Today I woke up thinking about how it would be possible to take the idea of -madScientistsDiary- forward. What would be the limitations and difficulties, how would it be possible to disseminate information about the work in progress... if that would be possible.

I was also preoccupied with issues such as newspaper formatting and everything else.

So, the possibility of a pilot project emerged, in which the practice would demonstrate the best path and the difficulties.

Just now I learned about the launch of the call for papers for the Symposium IOSTE 2022.

https://ioste.org/ioste\_brochure\_2020-2022.pdf

Given a number of factors that have converged in recent days, I decided to submit a proposal for this symposium.

At that point I understood, that dissemination can even start with an idea. Perhaps it is even recommended that this be so...

My idea for IOSTE: is it possible to unite research, collaborative work, information technology and transparency in the quest to improve undergraduate engineering education?

I will soon format this first report in the form of a newspaper article, with bibliographical references, abstract and whatever else is appropriate. I hope this will be the first article in -madScientistsDiary- and that it will bring good results and many other ideas.

O mais engraçado do texto que apresento aqui, é que dentro do texto surgiu de forma bem explícita o tema que será trabalhado no artigo do IOSTE. E acredito que esse pode mesmo ser o tema: a prática indica o caminho. 

Isse tópico está diretamente associado aos trabalhos que eu desenvolvi junto com alunos em meus projetos de extensão. Eu desenvolvi tais projetos, os planos de aula dos cursos que ministro, e tudo mais, sempre com esse enfoque: ensinar através da prática. E, é claro, isso está diretamente relacionado à pesquisa e à extensão. Mas nem sepre é fácil fazer isso. 

São 4:37 da manhã. Eu estou me sentindo muito enjoada. Possivelmente porque eu comi comida estragada na quermesse de uma capela próxima à minha casa. Em uma procissão para Virgem Maria. Eu me sinto muito enjoada, tenho dores fortes de cabeça, em especial na nuca, dores no corpo, em especial nos braços, também tenho muito enjôo e estou com diarréia. Isso tudo começou com minha curiosidade, a após eu ter comido os malditos cachorros quentes na feirinha da capela. Ah... com refrigerante, é claro. 

\end{document}

