\documentclass{book}
\usepackage[brazilian]{babel}
\usepackage[utf8]{inputenc}
\usepackage[T1]{fontenc}

\title{Preparation of an article for submission to IOSTE 2022}

\author{Tatiana Balbi Fraga$^{1*}$}

\begin{document}

Text under construction \\

Today I woke up thinking about how it would be possible to take the idea of -madScientistsDiary- forward. What would be the limitations and difficulties, how would it be possible to disseminate information about the work in progress... if that would be possible.

I was also preoccupied with issues such as newspaper formatting and everything else.

So, the possibility of a pilot project emerged, in which the practice would demonstrate the best path and the difficulties.

Just now I learned about the launch of the call for papers for the Symposium IOSTE 2022.

https://ioste.org/ioste\_brochure\_2020-2022.pdf

Given a number of factors that have converged in recent days, I decided to submit a proposal for this symposium.

At that point I understood, that dissemination can even start with an idea. Perhaps it is even recommended that this be so...

My idea for IOSTE: is it possible to unite research, collaborative work, information technology and transparency in the quest to improve undergraduate engineering education?

I will soon format this first report in the form of a newspaper article, with bibliographical references, abstract and whatever else is appropriate. I hope this will be the first article in -madScientistsDiary- and that it will bring good results and many other ideas.

The funniest part of the text I present here is that within the text the theme that will be worked on in the IOSTE article emerged very explicitly. And I believe that this could be the theme: practice shows the way.

This topic is directly associated with the work I developed with students in my extension projects. I developed such projects, the lesson plans of the courses I teach, and everything else, always with this focus: teaching through practice. And, of course, this is directly related to research and extension. But it's not always easy to do this.

It's 4:37 in the morning (when I write this new part of the text). I am feeling very nauseous. Possibly because I ate spoiled food at a chapel fair near my house. In a procession to Virgin Mary. I feel very nauseous, I have severe headaches, especially in the back of my neck, body aches, especially in my arms, I also have a lot of nausea and I have diarrhea. It all started with my curiosity just after I ate the damn hot dogs at the chapel fair. Ah... with soda, of course.

Anyway, I hope God forgives me and helps me get rid of this evil. I'm not Catholic. I was in the procession because I'm curious. And I ate the hot dogs because I wanted to help the church, I always loved to eat hot dogs, and they were delicious. But, it had been a long time since I had eaten bread and other industrialized products. I've been avoiding eating these things because I feel sick doing it.

And... starting a new day at work... today I need to correct the students' tests and assignments. Because I need to release the notes. Within a university, students must always be a priority.

I must thank google translator for helping me with the translations. It's amazing how google translations evolved over time. Today it's working almost perfectly :) 

\end{document}

